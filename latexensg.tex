%----------------------------------------------------------------
% FEUILLE DE STYLE ENSG au format Latex
% Création : sept. 2010 (D Lercier)
% Modification sept. 2012 (T Coupin)
%----------------------------------------------------------------

\documentclass{themeensg}

%---Texte en filigranne---
%pour l'enlever : \SetWatermarkText{}
%-------------------------

%---Mes packages à moi---
%\usepackage{}
%------------------------

%---Mes raccourcis---
\newcommand{\transpose}[1]{{}^t \! #1}
\newcommand{\ensg}{\textsc{Ensg}}
%--------------------

%---Paramètres du pdf---
    \hypersetup{
       backref=true,                           % Permet d'ajouter des liens dans
       pagebackref=true,                       % les bibliographies
       hyperindex=true,                        % Ajoute des liens dans les index.
       colorlinks=true,  %Colorise les liens : true pour version numérique, false pour version d'impression
       breaklinks=true,                        % Permet le retour à la ligne dans les liens trop longs.
       urlcolor= blue,                         % Couleur des hyperliens.
       linkcolor= blue,                       % Couleur des liens internes.
       bookmarks=true,                         % Créé des signets pour Acrobat.
       %bookmarksopen=true,                    % Si les signets Acrobat sont créés,
                                               % les afficher complètement.
       pdftitle={Thème ENSG},                 % Titre du document.
                                               % Informations apparaissant dans
       pdfauthor={Thibault Coupin},                      % dans les informations du document
       pdfsubject={Feuille de style ENSG}           % sous Acrobat.
    }

%-----------------------



%-------------------------------------------------------------

\setcounter{tocdepth}{1} %profondeur de la table des matières

\title{Feuille de style \LaTeX~de l'\ensg\\Documentation plus que rapide\\Version provisoire du \today~à \timenow}

%
%-------------------------------------------------------------
% Début du document
%--------------------------------------------------------------
\begin{document}
%--------------------------------------------------------------
\begin{titlepage}
%Inclusion des labels des entreprises
%Pour un seul label (à gauche), mettre NULL pour les 3e et 4e argument
\enterprise 
{logos/logo_ensg.jpg}
{Ecole Nationale des Sciences Géographiques}
{logos/logo_ensg.jpg}
{Ecole Nationale des Sciences Géographiques}

%Inclusion du titre
\maketitle{Stage de fin d'études \\Cycle des Ingénieurs diplômés de l'ENSG 3\up{ème} année }{logos/logo_ensg.jpg}

\infos{Mohamed-Amjad LASRI}{Septembre 2015}
\end{titlepage}


%---Page du jury---
%---Page du jury---
\newevenpage
\thispagestyle{plain}
\section*{Jury}
\vspace{0.5cm}

\textbf{Président de jury :} \\

Le président de jury

\vspace{0.5cm}

\textbf{Commanditaire :} \\

le commanditaire

\vspace{0.5cm}

\textbf{Encadrement de stage :} \\ 


qui a encadré ?

\vspace{0.5cm}

\textbf{Responsable pédagogique du cycle Ingénieur :} \\

Serge Botton, IGN/ENSG/DE/DPTS

\vspace{0.5cm}

\textbf{Tuteur du stage pluridisciplinaire :} \\

Patricia Parisi, IGN/ENSG/DE/DSHI

\vspace{1cm}

\copyright \hspace{0.3cm} ENSG

\section*{Stage de fin d'étude du xxx au xxx }
\vspace{0.3cm}
\textbf{Diffusion web :} $\boxtimes$ Internet \hspace{0.2cm}$\boxtimes$ Intranet Polytechnicum\hspace{0.2cm}
$\boxtimes$ Intranet ENSG\vspace{0.3cm}

\textbf{Situation du document :} 
\vspace{0.2cm}
\par
Rapport de stage de fin d'études présenté en fin de 3\up{ème} année du cycle des Ingénieurs
\vspace{0.3cm}


\newcounter{x}
\setcounter{x}{\getpagerefnumber{LastPage}-\getpagerefnumber{beginappendices}+1}

\textbf{Nombres de pages :} \getpagerefnumber{LastPage} pages dont \arabic{x} d'annexes
\vspace{0.3cm}

\textbf{Système hôte :} \LaTeX
\vspace{1cm}


\textbf{Modifications :} 
\begin{center}
\begin{tabular}{|c|c|c|>{\centering}p{6.5cm}|}
\hline 
EDITION & REVISION & DATE & PAGES MODIFIEES\tabularnewline
\hline
\hline 
1 & 0 & 09/2012 & Création\tabularnewline
\hline 

\end{tabular}
\end{center}
%------------------

%------------------------------------------------------------------------------
% Remerciements
%\newevenpage
\chapter*{Remerciements}

Je tiens à remercier toutes les personnes qui ont participé de différentes façons à la réussite de mon stage et plus particulièrement les personnes que je cite ci-dessous.

Olivier MARTIN, Frederic VERLUISE, Christian THOM et Christophe MEYNARD qui m'ont encadré, conseillé et ont répondu régulièrement à mes questions tout au long de mon stage.

Emmanuel BARDIERE, mon référent de stage ENSG, qui a suivi l'évolution de mon stage tout au long de ces cinq mois.

Tout le personnel du Laboratoire d'Opto-Électronique et de la société KYLIA.



%---Résumé (français)---
\begin{abstract}
\thispagestyle{empty}
	\vspace{1cm}

	Ceci est mon résumé
	
	\vspace{1.5cm}
	
	\textbf{Mots clés :} clés, clés, clés
\end{abstract}
%-----------------------


%---Résumé (anglais)---
\selectlanguage{english}
\begin{abstract}
\thispagestyle{empty}
	\vspace{1cm}
	
	This is my abstract
	
	\vspace{1.5cm}
	
	\textbf{Key words:} key, key, key
\end{abstract}
%----------------------

\selectlanguage{frenchb}

%---Table des matières, des figures et des tableaux---
%\newevenpage
\tableofcontents

\newevenpage
\listoffigures

\newevenpage
\listoftables
%----------------------------------------------------

%\newevenpage
\chapter*{Glossaire et sigles utiles}
\addcontentsline{toc}{chapter}{Glossaire et sigles utiles}

  \begin{acronym}
  \acro{ENSG}{\'Ecole Nationale des Sciences Géographiques}
  \acro{FIFO}{First In First Out}
  \acro{G3OS}{Geocube Operating System}
  \acro{GNSS}{Global Navigation Satellite Systems}
  \acro{GPS}{Global Positionning System}
  \acro{IEC}{The International Electrotechnical Commission}
  \acro{IHM}{Interface Homme Machine}
  \acro{ISO}{The International Organization for Standardization}
  \acro{NFC}{Near Field Communication}
  \acro{PCD}{Proximity Coupling Device}
  \acro{PICC}{Proximity Inductive Coupling Card}
  \acro{RTOS}{Real Time Operating System}
  \acro{TCP/IP}{Transmission Control Protocol/Internet Protocol}
  \end{acronym}


%---Introduction------------------------------------------------------------------
\newevenpage
\chapter*{Introduction}
  \addcontentsline{toc}{chapter}{Introduction}
  
  \vspace{1.5cm}
  
	La communication en champ proche (NFC) est une méthode de communication utilisée pour l'identification des personnes et des objets

%-------------------------------------------------------------------------------

\evenchapter[Concepts clés et problématiques]{Concepts clés et problématiques}

\textit{ Ce chapitre a pour but d'introduire le lecteur aux concepts clés nécessaires pour comprendre le travail effectué lors de ce stage.
Le lecteur est prié de prêter une attention particulière au tableau des sigles lors de la lecture du présent document. Les protocoles qu'on présente utilisent un certain nombre d’abréviations conventionnelles pour désigner Les opérations d'échanges entre. }

\section{Les Géocubes}
La miniaturisation des capteurs ainsi que la baisse des coûts de fabrication et de la consommation électrique des puces GNSS sont des facteurs qui peuvent laisser à envisager d'abandonner sur le terrain un réseau de capteurs opérant en permanence. Certains de ces capteurs GNSS permettent d'effectuer des mesures sur la phase donnant la possibilité de remonter à des précisions millimétriques, d'où l'idée d'un réseau de Géocubes. Le système Géocube est un réseau de capteurs GPS conçu et développé par le Laboratoire d'Opto-Éléctronique de Mesure et d'Instrumentation de l'Institut Géographique et Forestière Nationale. Il a comme objectif de mesurer les déformations avec une précision millimétrique. Ce réseau de capteurs a la particularité d'être très peu gourmand en énergie. On peut envisager de l'abandonner dans un milieu difficilement accessible sans qu'on ait à se soucier de son alimentation continue en électricité. En plus d'un module radio, un Géocube peut supporter plusieurs couches de capteurs lui permettant de collecter un certain nombre d'informations sur son environnement.

Dans les premières versions du Géocube initiés par LOEMI, Un opérateur humain peut communiquer directement avec un géocube en utilisant une de ces méthodes:

\begin{itemize}
\item Liaison série filaire, qui permet d'envoyer des commandes  à travers l'interface en lignes de commandes de G3OS.
\item Radio: Un protocol propriétaire particulier (DigiMesh\textcopyright) est utilisé pour communiquer avec un Géocube.
\end{itemize}

Dans la version industrielle du Géocube maintenue par la société Kylia, une attention particulière est prêtée à l'étanchéité du boîtier qui le contient et à la robustesse du produit final. Ce choix crucial se justifie principalement par le fait qu'un réseau de Géocube peut être destiné à la surveillance environnementale en temps de crise et doit, par conséquence, être résistant aux conditions extremes que peut présenter un tel contexte.

Un réseau de Géocubes communique à travers le protocol radio DigiMesh\textcopyright et l'utilise aussi pour centraliser les mesures acquises par les Géocubes vers un ordinateur, appelé coordinateur. Le coordinateur est le composant central du réseau, il encapsule toute la logique liée au traitement et au stockage des données. Il permet aussi à l'utilisateur de lancer des séries de calculs, récupérer les résultats ou de communiquer avec un Géocube en lançant des commandes qui seront transmises par radio. La Figure\ref{fig:geocube_network} résume ce schéma de fonctionnement.

\begin{figure}[h!]
\centering
\includegraphics[scale=0.8]{images/fig1.png}
\caption{Un réseau de Géocubes}
\label{fig:geocube_network}
\end{figure}

\section{Les systèmes embarqués et les noyaux temps-réel dur:}
Un système embarqué est par définition: Un système éléctronique et informatique autonome, souvent temps-réel, spécialisé dans une tâche bien précise []. De cette définition on peut ressortir deux éléments clés:
\begin{itemize}
\item Un système embarqué nécessite un développement matériel (électronique) mais aussi logiciel.
\item Un système embarqué a la particularité d'opérer en temps-réel.
\end{itemize}
Le lecteur peut se poser la question légitime: Pourquoi on insiste sur le "temps-réel" dans cette définition? Nos ordinateurs personnels n'opèrent-ils pas en "temps-réel"?. Pour répondre à ces questions et introduire l'importance de ... dans le cas du système Géocube Il faut comprendre que le concept de temps réel en informatique est très relatif et varie d'un métier à un autre: Le temps-réel pour un développeur web est de pouvoir fournir à l'internaute de l'information sous forme de flux, en tolérant les temps de latence qui peuvent résulter parfois des temps d'accès à une base de données ou à la bande passante d'internet. Pour un développeur qui fait de l'informatique pour automobiles et doit, par exemple, développer les couches logicielles relatives à un système d'airbag, Le temps-réel dans ce cas est très strict et la quantification de ce temps latence est primordiale, sinon la vie des gens serait en danger.
\subsection{Les tâches}
Une tâche est le composant principal d'un RTOS. Lorsque vous effectuez plusieurs tâches simultanées sur un ordinateur avec une mémoire vive limitée, vous pouvez remarquer à partir d'un certain ... que vos tâches auront du mal à tourner ... Pour les systèmes d'exploitation en temps-réel, communément connus sous le nom de RTOS (Real Time Operating Systems) La quantification de ce temps de latence est primordiale.

Dans cette perspective, une équipe de chercheurs du LOEMI ont mis au point un RTOS adapté aux tâches qui sont effectuées par un Géocube.

Une liste non exhaustive de ces tâches serait alors:

\begin{itemize}
\item Une tâche GPS qui gère toutes les opérations en relation avec l'acquisition des données GPS:
\item Une tâche Radio qui gère toutes les orpérations relatives à l'envoi et  à la réception des données et des commandes par radio;
\item Une tâche accéléromètre qui gère l'acquisition des données de l'accéleromètre...
\end{itemize}

On note ici qu'à chaque tâche on affecte une priorité. On revient à notre exemple d'airbag pour mieux appréhender cette notion. Imaginons maintenant que dans un RTOS destiné à l'industrie automobile on ne donne pas à la tâche qui gère l'airbag la plus haute priorité. Cela reviendrait à dire qu'à 

\subsection{La communication entre les tâches}

Dans un RTOS généralement, et dans G3OS plus particulièrement, une tâche peut communiquer avec ses semblables ou répondre à des signaux provenant des capteurs, qu'on appelle interruption.

Un signal d'interruption permet à un composant du système embarqué de notifier le micro-contrôleur central de l'arrivée d'un événement qui mérite son attention. Le choix de ces événements se fait souvent en programmant les registres des composants du système. Conventionnellement, le registre qui permet de choisir les événements déclencheurs d'interruptions s'appelle le registre principal des interruptions (Main Interrupt Register).

La communication entre les tâches s'effectue par plusieurs méthodes. La principale connue est la queue de messages. Ce mécanisme permet à une tâche de communiquer avec les autres en envoyant des messages. Un exemple typique serait alors: une tâche qui s'occupe de l'acquisition des données d'une puce GPS. Dès qu'une nouvelle trame de données est disponible, Cette tâche envoie  et une autre du traitement de ces données, Dans ce cas la première tâche envoie à la deuxième les données acquises à travers la que

\begin{figure}[h!]
\centering
\includegraphics[scale=1]{images/fig2.png}
\caption{Exemple multi tâches}
\label{fig:multitask_ex}
\end{figure}

\section{La communication en champs proche (NFC)}
La communication en champs proche est un ensemble de protocoles permettant d'établir une connexion radio entre deux dispositifs avec une distance ne dépassant pas 4cm. Aujourd'hui, on compte des millions d'objets connectés contenant la technologie NFC(cartes bancaires, smartphone, arrêts de bus, smartwatch...). L'interopérabilité entre les différentes puces équipant ces objets a poussé les constructeurs à mettre en place un certain nombre de normes régissant la fabrication, la programmation et l'utilisation de cette technologie.

La guerre des normes a fait converger les constructeurs vers l'ISO/IEC 14443. Cette norme encapsule en elle même quatre sous normes:

\begin{itemize}
\item ISO/IEC 14443-1: Description des couches physiques
\item ISO/IEC 14443-2: 
\item ISO/IEC 14443-3:
\item ISO/IEC 14443-4: 
\end{itemize}

\subsection{couches matérielles:}
La conception des couches matérielles d'un circuit NFC doit respecter les recommandations de l'ISO/IEC 14443-1 et une partie de l'ISO/IEC 14443-2 pour garantir l'interopérabilité avec les autres dispositifs disponibles sur le marché. Un circuit NFC typique est composé de 3 parties principales:
\begin{itemize}
\item Une antenne: Selon les spécifications de la dite norme les dimensions de l'antenne ne doivent pas excéder 86mm x 54mm x 3mm.
\item Une capacité adaptée pour garantir une raisonnance du circuit sur la fréquence 13.56Mhz;
\item Le PICC
\end{itemize}
\subsection{couches logicielles:}


\section{Les infrastructures de production des logiciels}
La conception et le développement des solutions logiciels dans un milieu industriel nécessite des infrastructures permettant d'automatiser un certain nombre de tâches qui, ensemble, forment ce qui est communément connu sous le nom de pipeline de production logicielle.

Cette chaîne de production est itérative, Elle favorise les cycles courts pour délivrer au client un produit évolutif et s'adaptant à ses besoins. Dans la Figure... on présente les principales étapes de cette chaîne.

Comme montré dans la Figure.... Une pipeline de production logicielle a un certain nombre d'acteurs externes humains qui garantissent son alimentation en versions(1) et en tickets(2). On définit alors ces acteurs comme suit:

\begin{itemize}
\item Développeur: s'occupe de la conception et le développement des solutions informatiques en réponse aux besoins des clients exp.. dans le gestionnaire des bugs. Son travail permet d'alimenter le gestionnaire de versions.
\item Product owner: terme emprunté à la méthode Scrum. Il est l'interlocuteur unique des clients et permet de traduire leurs besoins en tickets(Gestionnaire de tickets)
\item Testeur: 
\end{itemize}
En plus de ces acteurs humains une infrastructure de production logicielle contient des composants logiciels pour automatiser un certain nombre de tâches:
\begin{itemize}
\item Gestionnaire de bugs: Comme son nom l'indique ce composant est un outil de communication et de traçabilité permettant de suivre l'évolution de la réponse du développeur au besoin du client et à la correction des bugs.
\item Gestionnaire de versions: Cet outil est un classique de la gestion des projets informatiques, il permet, entre autres, aux développeurs de collaborer sur le même code source sans que cela n'affecte  d'archiver tous les changements effectués sur un code source. Par souci de traçabilité, un gestionnaire de version est indispensable dans un projet informatique même s'il n y a qu'un développeur.
\item Intégration continue
\end{itemize}



\begin{comment}
\begin{itemize}
\item \texttt{themeensg.cls} : contient les personnalisations et macros utiles
\item \texttt{jury.tex} : pour la feuille de présentation du jury
\item le dossier \texttt{images} : il doit contenir toutes les images, il contient déjà le dossier logo avec celui de l'\ensg
\item \texttt{bibliographie.bib} contient la bibliographie
\end{itemize}
\end{comment}
\section{Commandes personnalisées}

\begin{itemize}
\item \verb!\newevenpage! : identique à \verb!\newpage! mais en insère une page blanche de façon à débuter la nouvelle page sur un numéro de page impaire.
\item \verb!\evenchapter{titre}! : démarre un nouveau chapitre sur une page impaire,\\ \verb!\evenchapter[titre sommaire]{titre}! fonctionne aussi mais pas \verb!\evenchapter*{titre}!
\item idem pour \verb!\evenpart{titre}!
\end{itemize}

\section{Fichier source de cette doc}
Ce fichier \texttt{tex} contient toute la structure d'un rapport mais une bonne partie est désactivée car commentée par l'environnement \verb!\begin{comment} ... \end{comment}!


%-------------------------------------------------------------------------------
\newevenpage
\chapter*{Conclusion}
  \addcontentsline{toc}{part}{Conclusion}
  \vspace{1.5cm}
Il est l'heure de conclure : bonne nuit !


%-------------------------------------------------------------------------------
% Insertion de la bibliographie
\newevenpage
\nocite{*}
\bibliographystyle{apalike}
\bibliography{bibliographie}

\newevenpage
\begin{appendices} 
\label{beginappendices}
\annexe[Filtre de Kalman]{Filtre\newline de Kalman}
\label{annexekalman}
Annexe 1

\end{appendices} 

\end{document}