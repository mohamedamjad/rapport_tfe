%----------------------------------------------------------------
% FEUILLE DE STYLE ENSG au format Latex
% Création : sept. 2010 (D Lercier)
% Modification sept. 2012 (T Coupin)
%----------------------------------------------------------------

\documentclass{themeensg}

%---Texte en filigranne---
%pour l'enlever : \SetWatermarkText{}
%-------------------------

%---Mes packages à moi---
%\usepackage{}
%------------------------

%---Mes raccourcis---
\newcommand{\transpose}[1]{{}^t \! #1}
\newcommand{\ensg}{\textsc{Ensg}}
%--------------------

%---Paramètres du pdf---
    \hypersetup{
       backref=true,                           % Permet d'ajouter des liens dans
       pagebackref=true,                       % les bibliographies
       hyperindex=true,                        % Ajoute des liens dans les index.
       colorlinks=true,  %Colorise les liens : true pour version numérique, false pour version d'impression
       breaklinks=true,                        % Permet le retour à la ligne dans les liens trop longs.
       urlcolor= blue,                         % Couleur des hyperliens.
       linkcolor= blue,                       % Couleur des liens internes.
       bookmarks=true,                         % Créé des signets pour Acrobat.
       %bookmarksopen=true,                    % Si les signets Acrobat sont créés,
                                               % les afficher complètement.
       pdftitle={Thème ENSG},                 % Titre du document.
                                               % Informations apparaissant dans
       pdfauthor={Thibault Coupin},                      % dans les informations du document
       pdfsubject={Feuille de style ENSG}           % sous Acrobat.
    }

%-----------------------



%-------------------------------------------------------------

\setcounter{tocdepth}{1} %profondeur de la table des matières

\title{Feuille de style \LaTeX~de l'\ensg\\Documentation plus que rapide\\Version provisoire du \today~à \timenow}

%
%-------------------------------------------------------------
% Début du document
%--------------------------------------------------------------
\begin{document}
%--------------------------------------------------------------
\begin{titlepage}
%Inclusion des labels des entreprises
%Pour un seul label (à gauche), mettre NULL pour les 3e et 4e argument
\enterprise 
{logos/logo_ensg.jpg}
{Ecole Nationale des Sciences Géographiques}
{logos/logo_ensg.jpg}
{Ecole Nationale des Sciences Géographiques}

%Inclusion du titre
\maketitle{Stage de fin d'études \\Cycle des Ingénieurs diplômés de l'ENSG 3\up{ème} année }{logos/logo_ensg.jpg}

\infos{Mohamed-Amjad LASRI}{Septembre 2015}
\end{titlepage}


%---Page du jury---
%---Page du jury---
\newevenpage
\thispagestyle{plain}
\section*{Jury}
\vspace{0.5cm}

\textbf{Président de jury :} \\

Le président de jury

\vspace{0.5cm}

\textbf{Commanditaire :} \\

le commanditaire

\vspace{0.5cm}

\textbf{Encadrement de stage :} \\ 


qui a encadré ?

\vspace{0.5cm}

\textbf{Responsable pédagogique du cycle Ingénieur :} \\

Serge Botton, IGN/ENSG/DE/DPTS

\vspace{0.5cm}

\textbf{Tuteur du stage pluridisciplinaire :} \\

Patricia Parisi, IGN/ENSG/DE/DSHI

\vspace{1cm}

\copyright \hspace{0.3cm} ENSG

\section*{Stage de fin d'étude du xxx au xxx }
\vspace{0.3cm}
\textbf{Diffusion web :} $\boxtimes$ Internet \hspace{0.2cm}$\boxtimes$ Intranet Polytechnicum\hspace{0.2cm}
$\boxtimes$ Intranet ENSG\vspace{0.3cm}

\textbf{Situation du document :} 
\vspace{0.2cm}
\par
Rapport de stage de fin d'études présenté en fin de 3\up{ème} année du cycle des Ingénieurs
\vspace{0.3cm}


\newcounter{x}
\setcounter{x}{\getpagerefnumber{LastPage}-\getpagerefnumber{beginappendices}+1}

\textbf{Nombres de pages :} \getpagerefnumber{LastPage} pages dont \arabic{x} d'annexes
\vspace{0.3cm}

\textbf{Système hôte :} \LaTeX
\vspace{1cm}


\textbf{Modifications :} 
\begin{center}
\begin{tabular}{|c|c|c|>{\centering}p{6.5cm}|}
\hline 
EDITION & REVISION & DATE & PAGES MODIFIEES\tabularnewline
\hline
\hline 
1 & 0 & 09/2012 & Création\tabularnewline
\hline 

\end{tabular}
\end{center}
%------------------

%------------------------------------------------------------------------------
% Remerciements
\newevenpage
\chapter*{Remerciements}

Je tiens à remercier toutes les personnes qui ont participé de différentes façons à la réussite de mon stage et plus particulièrement les personnes que je cite ci-dessous.

Olivier Martin, Frederic VERLUISE, Christian THOM et Christophe MEYNARD qui m'ont encadré, conseillé et ont répondu régulièrement à mes questions tout au long de mon stage.

Emmanuel BARDIERE, mon référent de stage ENSG, qui a suivi l'évolution de mon stage tout au long de ces cinq mois.

Tout le personnel du Laboratoire d'Opto-Électronique et de la société KYLIA.



%---Résumé (français)---
\begin{abstract}
\thispagestyle{empty}
	\vspace{1cm}

	Ceci est mon résumé
	
	\vspace{1.5cm}
	
	\textbf{Mots clés :} clés, clés, clés
\end{abstract}
%-----------------------


%---Résumé (anglais)---
\selectlanguage{english}
\begin{abstract}
\thispagestyle{empty}
	\vspace{1cm}
	
	This is my abstract
	
	\vspace{1.5cm}
	
	\textbf{Key words:} key, key, key
\end{abstract}
%----------------------

\selectlanguage{frenchb}

%---Table des matières, des figures et des tableaux---
\newevenpage
\tableofcontents

\newevenpage
\listoffigures

\newevenpage
\listoftables
%----------------------------------------------------

\newevenpage
\chapter*{Glossaire et sigles utiles}
\addcontentsline{toc}{chapter}{Glossaire et sigles utiles}

  \begin{acronym}
  \acro{ENSG}{\'Ecole Nationale des Sciences Géographiques}
  \acro{GNSS}{Global Navigation Satellite Systems}
  \acro{GPS}{Global Positionning System}
  \end{acronym}


%---Introduction------------------------------------------------------------------
\newevenpage
\chapter*{Introduction}
  \addcontentsline{toc}{chapter}{Introduction}
  
  \vspace{1.5cm}
	J'introduis

%-------------------------------------------------------------------------------

\evenchapter[La nouvelle feuille de style \ensg]{La nouvelle feuille\newline de style \ensg}

\textit{Dans tout ce qui suit, et sauf mention contraire l'unité de temps utilisée est le ms. \ensg. }

\section{Les fichiers}
\begin{itemize}
\item \texttt{themeensg.cls} : contient les personnalisations et macros utiles
\item \texttt{jury.tex} : pour la feuille de présentation du jury
\item le dossier \texttt{images} : il doit contenir toutes les images, il contient déjà le dossier logo avec celui de l'\ensg
\item \texttt{bibliographie.bib} contient la bibliographie
\end{itemize}

\section{Commandes personnalisées}

\begin{itemize}
\item \verb!\newevenpage! : identique à \verb!\newpage! mais en insère une page blanche de façon à débuter la nouvelle page sur un numéro de page impaire.
\item \verb!\evenchapter{titre}! : démarre un nouveau chapitre sur une page impaire,\\ \verb!\evenchapter[titre sommaire]{titre}! fonctionne aussi mais pas \verb!\evenchapter*{titre}!
\item idem pour \verb!\evenpart{titre}!
\end{itemize}

\section{Fichier source de cette doc}
Ce fichier \texttt{tex} contient toute la structure d'un rapport mais une bonne partie est désactivée car commentée par l'environnement \verb!\begin{comment} ... \end{comment}!


%-------------------------------------------------------------------------------
\newevenpage
\chapter*{Conclusion}
  \addcontentsline{toc}{part}{Conclusion}
  \vspace{1.5cm}
Il est l'heure de conclure : bonne nuit !


%-------------------------------------------------------------------------------
% Insertion de la bibliographie
\newevenpage
\nocite{*}
\bibliographystyle{apalike}
\bibliography{bibliographie}

\newevenpage
\begin{appendices} 
\label{beginappendices}
\annexe[Filtre de Kalman]{Filtre\newline de Kalman}
\label{annexekalman}
Annexe 1

\end{appendices} 

\end{document}